\documentclass{article}
\usepackage[utf8]{inputenc}
\usepackage[spanish,mexico]{babel}
\usepackage[margin=2.3cm]{geometry}
\usepackage{graphicx}
\usepackage[export]{adjustbox}
\usepackage{caption}
\usepackage{subcaption}
\usepackage{fancyhdr}
\usepackage{lipsum}
\usepackage{float}
\usepackage{amsfonts, amsthm}
\usepackage{upgreek}
\usepackage{physics}
\usepackage{cancel}
\usepackage{amssymb, latexsym, amsmath}
\usepackage{amsthm}

% Notas al pie.
\renewcommand{\thefootnote}{\Roman{footnote}}

% Evitar sangrías
\setlength{\parindent}{0cm}

% Letter colors.
\usepackage[usenames,dvipsnames,svgnames,table]{xcolor}

% Código
\newcommand{\code}[1]{\textcolor{white!25!black}{\texttt{#1}}}
\usepackage{listings}

% Def. Dr. César.
\usepackage{tikz}
\usetikzlibrary{shapes,calc}
\tikzstyle{edge}=[shorten <=2pt, shorten >=2pt, >=stealth, line width=1.1pt]
\tikzstyle{blueE}=[shorten <=2pt, shorten >=2pt, >=stealth, line width=1.5pt, blue]
\tikzstyle{blackV}=[circle, fill=black, minimum size=6pt, inner sep=0pt, outer sep=0pt]
\tikzstyle{blueV}=[circle, fill=blue, draw, minimum size=6pt, line width=0.75pt, inner sep=0pt, outer sep=0pt]
\tikzstyle{redV}=[circle, fill=red, draw, minimum size=6pt, line width=0.75pt, inner sep=0pt, outer sep=0pt]
\tikzstyle{redSV}=[semicircle, fill=red, minimum size=3pt, inner sep=0pt, outer sep=0pt, rotate=225]
\tikzstyle{blueSV}=[semicircle, fill=blue, minimum size=3pt, inner sep=0pt, outer sep=0pt, rotate=225]
\tikzstyle{blackSV}=[semicircle, fill=black, minimum size=3pt, inner sep=0pt, outer sep=0pt, rotate=225]
\tikzstyle{vertex}=[circle, draw, minimum size=6pt, line width=0.75pt, inner sep=0pt, outer sep=0pt]

% Proof
\renewcommand*{\proofname}{\textbf{Soluci\'on:}}

% Margins
%\addtolength{\textheight}{3cm}

\usepackage{amsmath}
\usepackage{amsmath}
\usepackage{amssymb}
\usepackage{amsthm}
\usepackage{url}
\usepackage{hyperref}

%%Tablas con color
\usepackage{xcolor, colortbl}
\usepackage{array, multirow, multicol}
\cellcolor[modelo color]{color}

\pagestyle{fancyplain}

\title{Facultad de Ciencias, UNAM}
\date{\today}

\hypersetup{
  colorlinks=true,
  linkcolor=blue,
  filecolor=magenta,      
  urlcolor=cyan,
  pdftitle={Overleaf Example},
  pdfpagemode=FullScreen,
}

\begin{document}
\fancyhead[r]{ Computación Distribuida 2023-1}
\thispagestyle{empty}

\begin{figure}[ht]
  \minipage{0.76\textwidth}
  \includegraphics[width=4cm]{Logo_UNAM.png}
  \label{EscudoUNAM}
  \endminipage
  \minipage{0.32\textwidth}
  \includegraphics[height = 4.9cm ,width=4cm]{Logo_FC.png}
  \label{EscudoFC}
  \endminipage
\end{figure}

\begin{center}
  \vspace{0.8cm}
  \LARGE
  UNIVERSIDAD NACIONAL AUTÓNOMA DE MÉXICO 
  
  \vspace{0.7cm}
  \LARGE
  FACULTAD DE CIENCIAS
  
  \vspace{0.8 cm}	
  \Large
  \textbf{Tarea 2}

  \vspace{0.8 cm}
  \normalsize	
  INTEGRANTES \\
  \vspace{.2cm}
  \large
  \textbf{Torres Valencia Kevin Jair - \texttt{318331818}}\\
  \textbf{Aguilera Moreno Adrián - \texttt{421005200}}\\
  \textbf{Natalia Abigail Pérez Romero  - \texttt{318144265}}\\
  %\textbf{Nombre - \texttt{Número de cuenta}}
  
  \vspace{1 cm}
  \normalsize	
  PROFESOR \\
  \vspace{.2cm}
  \large
  \textbf{Miguel Ángel Piña Avelino}
  
  \vspace{1 cm}
  AYUDANTE \\
  \vspace{.2cm}
  \large
  \textbf{Pablo Gerardo González López}
  \vspace{1.3cm}
  
  \normalsize	
  ASIGNATURA \\
  \vspace{.2cm}
  \large
  \textbf{Computación Distribuida}
  
  \vspace{1 cm}
  \today
\end{center}


\newpage

Los \textbf{relojes vectoriales} son un tipo de
reloj lógico propuesto de manera independiente por
\textit{Colin J. Fidge} y \textit{Friedemann Mattern}
en 1988. 
Esta técnica consiste en un mapeo entre eventos en una
historia distribuida y vectores enteros.
De manera que nos permite determinar si eventos esta casualmente 
relacionados o no (si son concurrrentes). 
Un \textbf{sistema vectorial de relojes}. Es un mecanismo capaz 
de caracterizar estados locales (en adelante, eventos) en un sistema
distribuido, asociando un valor vectorial a cada estado.
El \textbf{tiempo vectorial} es la noción de tiempo capturada
por los relojes vectoriales.

\textbf{Caracterización Formal del Tiempo Vectorial.} Sea
\code{date(e)} la caracterización asociada a un evento \code{e},
de tal manera que se cumple:
\begin{enumerate}
\item $\forall_{e_1, e_2}:\left(e_1 \rightarrow e_2\right)
  \Leftrightarrow \code{date($e_1$)} < \code{date($e_2$)}$.
\item $\forall_{e_1, e_2}: \left(e_1\ ||\ e_2\right) \Leftrightarrow
  \code{date($e_1$)}\ ||\ \code{date($e_2$)}$.
\end{enumerate}

\textbf{Reloj Vectorial.} La implementación del tiempo
    vectorial requiere que cada proceso, en el sistema,
    mantenga un vector de enteros positivos $Vc_i[1, \dotsm, n]$
    con valores inicialmente $[0, \dotsm, 0]$. Este vector
    debe cumplir con
    \begin{enumerate}
    \item $Vc_i[i]$ cuenta el número de eventos producidos por
      $p_i$.
    \item $Vc_i[j]$, $j \not= i$, nos dice cuántos eventos conoce
      $p_i$ producido por $p_j$.
    \end{enumerate}
    De manera formal, sea $e$ un evento producido por $p_i$,
    tenemos que
    \[Vc_i[k] = \left|\{f | (f \text{ se produjo por } p_k) \land
    (f \rightarrow e)\}\right| + 1(k, i).\]


    \textbf{Evento relevante} En algún nivel de abstracción, sólo un subconjunto de los eventos son relevantes. Dado un computo distribuido
    $\widehat{H}=(H,\xrightarrow[]{ev})$, sea $R\subset H$ los eventos relevantes. Sin perder generalidad, consideramos que $R$ consiste solo de eventos internos.
    \textbf{El problema del seguimiento del predecesor inmediato (Immediate predecessor tracking IPT)} consiste en asociar a cada evento relevante el conjunto de eventos relevantes que son sus predecesores inmediatos. Además, esto se ha hecho sobre la marcha y sin añadir mensajes de control. La determinación de los predecesores inmediatos consiste en calcular la reducción transitiva (o diagrama de Hasse) del orden parcial $\widehat{R}=(R,\xrightarrow[]{re})$. 

    \textbf{Relojes vectoriales dinamicos}A menudo el número de procesos participando en un computo distribuido no es constante, por lo que los relojes debe ser capaces de crecer. Para alcanzar esta flexibilidad el vector de reloj es una matriz de dos columnas, variable en su número de filas, que proporciona un mapeo simple de la ID de un proceso al valor de reloj asociado (escalar).

\end{document}
