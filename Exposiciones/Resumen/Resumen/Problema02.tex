%%%%%%%%%%% Aquí va la solución al problema 1.
\newpage
\textbf{\textcolor{MidnightBlue}{2.}} Retomando el
problema de los dos enamorados con los mismos
requerimientos vistos en clase, responda las
siguientes preguntas:

\begin{itemize}
\item Suponga que las citas sólo se pueden realizar
  entre las 21:00 y las 22:00 horas. ¿Tiene solución
  el problema en este caso?
  
  El problema no tiene solución, porque a pesar de que
  el intervalo de tiempo en que la cita puede se realizada
  se vio reducido, persiste el problema de no poder reconocer
  cuando un mensaje ya ha sido entregado o se perdió, de manera
  que aún no es posible que los enamorados puedan ponerse de
  acuerdo para su cita.

\item ¿El problema tiene solución cuando añade el siguiente
  requerimiento: los amantes deben ser capaces de coordinar
  una hora para una cita solamente cuando ningún mensaje
  se pierde, y, en cualquier otro caso, ellos no deberían
  presentarse?
  
 El problema no tiene solución, la imposibilidad es justo igual
 a la del problema original. Los enamorados no pueden saber si
 su mensaje ha sido recibido y si deben presentarse a la cita.
 Podría suceder que los mensajes se pierdan continuamente, de
 forma que los enamorados nunca se reunan.

\item Consideremos una variación: Los dos amantes se han
  cuenta de que no necesitan ponerse de acuerdo sobre una
  hora exacta para la reunión, está bien si sus horas de
  reunión son lo suficientemente cercanas. En otras palabras,
  cada uno debería eventualmente elegir un tiempo, de modo
  que los dos tiempos estén lo suficientemente cerca. ¿Se
  puede resolver su problema?
  
  No, el problema no tiene solución. Porque es poco probable
  que dos intervalos sobre un periodo de tiempo indeterminado
  coincidan, además de que persiste el problema de indistinguibilidad,
  de forma que los enamorados tienen la incertidumbre si su mensaje
  fue recibido o no.
  
\end{itemize}
