%%%%%%%%%%% Aquí va la solución al problema 4.
\newpage
\textbf{\textcolor{MidnightBlue}{4.}} Considera un sistema
distribuido con $n \geq 2$ procesos, $p_1, p_2, \dotsm , p_n$,
en el que la gráfica de comunicación es la completa $K_n$.
El sistema es síncrono pero la comunicación no es confiable;
sea P el conjunto de todos los procesos que envían mensajes en
el tiempo $d$; entonces, hay dos posibilidades, todos los
mensajes de P llegan a su destino en el tiempo $d + 1$, o uno
de ellos se pierde y nunca llega a sus destino y los otros
en $P$ si llegan en el tiempo $d + 1$.
\newline

Considera un algoritmo A en el que cada proceso $p_i$ tiene como
entrada un identificador $ID_i$, que es un número natural (diferente
al de los demás), y cada proceso $p_i$ simplemente envía su $ID_i$
a los otros $n - 1$ procesos.
\newline

Dibuja cuales son todos los estados \textit{globales} posibles
(mundos posibles) en el tiempo 1 (los procesos mandan sus mensajes
en el tiempo 0). En cada estado global, especifica el estado
\textit{local} de cada proceso, es decir, la información que cada
proceso tiene en ese estado global; y entre cada par de estados
globales pinta una arista con los procesos que no pueden \textit{distinguir}
entre esos estados. ¿Es posible que cada proceso elija consistentemente
uno de los $ID$s de entre los que recibió de forma tal que en cada estado
global todos los procesos eligen el mismo $ID$? Argumenta tu respuesta.
\newline

\textbf{\textit{Solución:}}
Supongamos que $n = 3$, entonces la representación gráfica de lo requerido
se vería como el siguiente gráfico:


\begin{figure}[ht!]
  \centering
  \begin{tikzpicture}
  %%%%%%%%%%%%%%%%%% Nodos p_i
    \node (1)  [vertex, label=270:$\underbrace{p_1}_{\{ID_2, ID_3\}}$] at (0,0){};
    \node (2)  [redV, label=270:$\underbrace{p_2}_{\{ID_1\}}$] at (2,0){};
    \node (3)  [vertex, label=90:$\overbrace{p_3}^{\{ID_1, ID_2\}}$] at (1,1.2){};
    % Sin aristas
    \node (13)  [vertex, label=270:$\underbrace{p_1}_{\{ID_2, ID_3\}}$] at (-1,-4.2){};
    \node (14)  [redV, label=270:$\underbrace{p_2}_{\{ID_3\}}$] at (1,-4.2){};
    \node (15)  [vertex, label=90:$\overbrace{p_3}^{\{ID_1, ID_2\}}$] at (0,-3){};
    
    \node (4)  [vertex, label=90:$\underbrace{p_3}^{ID_1, ID_2}$] at  (9,1.2){};
    \node (5)  [blueV, label=270:$\underbrace{p_1}_{\{ID_2\}}$] at (8,0){};
    \node (6)  [vertex, label=270:$\underbrace{p_2}_{\{ID_1, ID_3\}}$] at (10,0){};
    % Sin aristas
    \node (16)  [vertex, label=90:$\overbrace{p_3}^{\{ID_1, ID_2\}}$] at  (10,-3){};
    \node (17)  [blueV, label=270:$\underbrace{p_1}_{\{ID_3\}}$] at (9,-4.2){};
    \node (18)  [vertex, label=270:$\underbrace{p_2}_{\{ID_1, ID_3\}}$] at (11,-4.2){};

    \node (7) [vertex, label=90:$\overbrace{p_1}^{\{ID_2, ID_3\}} $]  at (4,-2){};
    \node (8) [vertex, label=90:$\overbrace{p_2}^{\{ID_1, ID_3\}}$]   at (6,-2){};
    \node (9) [vertex, label=270:$\underbrace{p_3}_{\{ID_1, ID_2\}}$] at (5,-3.2){};

    \node (10) [blackV, label=90:$\overbrace{p_{3}}^{\{ID_1\}}$]         at (3, -6.2){};
    \node (11) [vertex, label=270:$\underbrace{p_{1}}_{\{ID_2, ID_3\}}$] at (4, -7.4){};
    \node (12) [vertex, label=270:$\underbrace{p_{2}}_{\{ID_1, ID_3\}}$] at (2, -7.4){};

    % Sin aristas
    \node (19) [blackV, label=90:$\overbrace{p_{3}}^{\{ID_2\}}$]         at (7, -6.2){};
    \node (20) [vertex, label=270:$\underbrace{p_{1}}_{\{ID_2, ID_3\}}$] at (8, -7.4){};
    \node (21) [vertex, label=270:$\underbrace{p_{2}}_{\{ID_1, ID_3\}}$] at (6, -7.4){};

    %%%%%%%%%%%%%%%%%%%%%% Aristas
    \foreach \i in {1,2}
    \draw [edge] let \n1={3} in (\i) to (\n1);
    \foreach \i in {5,6}
    \draw [edge] let \n1={4} in (\i) to (\n1);
    \foreach \i in {7,8}
    \draw [edge] let \n1={9} in (\i) to (\n1);
    \foreach \i in {11,12}
    \draw [edge] let \n1={10} in (\i) to (\n1);
    \foreach \i in {13,14}
    \draw [edge] let \n1={15} in (\i) to (\n1);
    \foreach \i in {17,18}
    \draw [edge] let \n1={16} in (\i) to (\n1);
    \foreach \i in {20,21}
    \draw [edge] let \n1={19} in (\i) to (\n1);
    \draw [edge]  (1) to (2);
    \draw [edge]  (5) to (6);
    \draw [edge]  (7) to (8);
    \draw [edge]  (11) to (12);
    \draw [edge]  (20) to (21);
    \draw [edge]  (17) to (18);
    \draw [edge]  (13) to (14);
    
    \node (L) at (-1,2){\textbf{P}};
    \draw [edge]  (-2,3) to (12,3);
    \draw [edge]  (-2,-9) to (12,-9);
    \draw [edge]  (-2,3) to (-2,-9);
    \draw [edge]  (12,3) to (12,-9);
  \end{tikzpicture}
\end{figure}

donde, a cada $p_i$ ($1 \leq i \leq 3$) le corresponde un $ID_i$ de
tal manera que para cada estado global se identifica si todos sus
procesos fueron recibidos de manera exitosa, en caso de no ser así
se colorean.
Así, por cada proceso, tenemos que
\begin{enumerate}
\item Sabemos que, a lo más, un mensaje no le llegó a $p_1$,
o no le llegó a $p_2$, o no le llegó a $p_3$ (de manera exclusiva).
\item También sabemos que en el tiempo $0$ cada proceso envía
un identificador $ID_i$ al resto de sus vecinos, entonces,
para un proceso $p_i$ que no recibió un $ID$ tenemos que existen
dos posibles casos\footnote{Cada proceso tiene dos vecinos, por trabajar
con $K_3$, en este caso partícular.}.
\end{enumerate}

\textbf{Obs.} Los subíndices superiores o inferiores en cada gráfica
nos dice que identificadores $ID_i$ han recibido\footnote{El estado local
del proceso.}. Además, se obvia que el proceso $p_i$ ya tiene en su
estado local al identificador $ID_i$.
\newline

A continuación mostramos un subconjutnto  $A \subseteq P$, tal que cada proceso
en el que no se pude distinguir, dentro del estado global, se conecten entre
ellos

\begin{figure}[ht!]
  \centering
  \begin{tikzpicture}
  %%%%%%%%%%%%%%%%%% Nodos p_i
    \node (1)  [vertex, label=270:$p_1$] at (0,0){};
    \node (2)  [redV, label=90:$\overbrace{p_2}^{\{ID_1\}}$] at (2,0){};
    \node (3)  [vertex, label=90:$p_3$] at (1,1.2){};
    % Sin aristas
    \node (13)  [vertex, label=270:$p_1$] at (-1,-4.2){};
    \node (14)  [redV, label=270:$\underbrace{p_2}_{\{ID_3\}}$] at (1,-4.2){};
    \node (15)  [vertex, label=90:$p_3$] at (0,-3){};
    
    \node (4)  [vertex, label=90:$p_3$] at  (9,1.2){};
    \node (5)  [blueV, label=90:$\overbrace{p_1}^{\{ID_2\}}$] at (8,0){};
    \node (6)  [vertex, label=270:$p_2$] at (10,0){};
    % Sin aristas
    \node (16)  [vertex, label=90:$p_3$] at  (10,-3){};
    \node (17)  [blueV, label=270:$\underbrace{p_1}_{\{ID_3\}}$] at (9,-4.2){};
    \node (18)  [vertex, label=270:$p_2$] at (11,-4.2){};

    \node (10) [blackV, label=180:$\underbrace{p_{3}}_{\{ID_1\}}$] at (3, -6.2){};
    \node (11) [vertex, label=270:$p_{1}$] at (4, -7.4){};
    \node (12) [vertex, label=270:$p_{2}$] at (2, -7.4){};

    % Sin aristas
    \node (19) [blackV, label=360:$\underbrace{p_{3}}_{ID_2}$] at (7, -6.2){};
    \node (20) [vertex, label=270:$p_{1}$] at (8, -7.4){};
    \node (21) [vertex, label=270:$p_{2}$] at (6, -7.4){};

    %%%%%%%%%%%%%%%%%%%%%% Aristas
    \foreach \i in {1,2}
    \draw [edge] let \n1={3} in (\i) to (\n1);
    \foreach \i in {5,6}
    \draw [edge] let \n1={4} in (\i) to (\n1);
    \foreach \i in {11,12}
    \draw [edge] let \n1={10} in (\i) to (\n1);
    \foreach \i in {13,14}
    \draw [edge] let \n1={15} in (\i) to (\n1);
    \foreach \i in {17,18}
    \draw [edge] let \n1={16} in (\i) to (\n1);
    \foreach \i in {20,21}
    \draw [edge] let \n1={19} in (\i) to (\n1);
    \draw [edge]  (1) to (2);
    \draw [edge]  (5) to (6);
    \draw [edge]  (11) to (12);
    \draw [edge]  (20) to (21);
    \draw [edge]  (17) to (18);
    \draw [edge]  (13) to (14);
    
    \node (L) at (-1,2){\textbf{A}};
    \draw [edge]  (-2,3) to (12,3);
    \draw [edge]  (-2,-9) to (12,-9);
    \draw [edge]  (-2,3) to (-2,-9);
    \draw [edge]  (12,3) to (12,-9);

    % Conexiones
    \draw [edge]  (2) to (14);
    \draw [edge]  (2) to (5);
    \draw [edge]  (2) to (17);
    \draw [edge]  (2) to (10);
    \draw [edge]  (2) to (17);
    \draw [edge]  (2) to (19);
    \draw [edge]  (5) to (14);
    \draw [edge]  (5) to (17);
    \draw [edge]  (5) to (10);
    \draw [edge]  (5) to (19);
    \draw [edge]  (10) to (19);
    \draw [edge]  (10) to (14);
    \draw [edge]  (10) to (17);
    \draw [edge]  (14) to (17);
    \draw [edge]  (19) to (14);
    \draw [edge]  (19) to (17);
  \end{tikzpicture}
\end{figure}
podemos observar que se ha ignorado la información no relevante para
los procesos que no puede distinguir entre los demás estados.
\newline

Por último, demos respuesta a la pregunta: ¿Es posible que cada
proceso elija consistentemente uno de los $ID$s de entre los que
recibió de forma tal que en cada estado global todos los procesos
eligen el mismo $ID$?

\textbf{R./} No, esto es \textit{falso} en general. Supongamos que en cada
estado podemos elegir el mismo $ID_i$ para todos los procesos, basta con
ver que pasa para algún estado global en donde un proceso $p_i$ no recibe
exactamente el identificador $ID_j$\footnote{Este estado vive en $P$.}
con $j \not= i$ y $1 \leq j \leq 3$.

Caso partícular con $n = 3$, es que todos los procesos eligan a $ID_2$
de manera consistente, claramente el estado global izquierdo superior no
cumplirá esto, pues $p_2$ no ha recibido este $ID_2$. En este caso, hay
más de un estado global que cumple no poder elegir de manera consistente
este $ID$. Como hay al menos un proceso $p_i$ en algún estado global
que no recibió un $ID_i$, y esto pasa para cada $ID_i$, la respuesta
a la pregunta es no.

Si solo existiera el caso ideal, donde todos los mensajes de $P$ llegan
a su destino, entonces la respuesta a esta pregunta sería sí. Bajo esta
condición y bajo la consideración de que haya, al menos, un $ID_i$ que
haya logrado ser recibido por todos los procesos (exceptuando el proceso
que envia), la respuesta sería sí.
\hfill $\lhd$
