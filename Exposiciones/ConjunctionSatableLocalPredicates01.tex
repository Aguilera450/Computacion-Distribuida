%%%%%%%%%%%%%%%% Conjunción de predicados estables.
\begin{frame}[fragile]{Aplicación:}{Detección de conjunciones de predicados estables.}
  \justifying
  \textbf{Def.} Un predicado es local para $p_i$ \textit{\textbf{Sii}}
  se encuentra en variables de $p_i$ solamente.
  
  \textbf{Def.} El predicado $LP_i$ es estable \textit{\textbf{Sii}}
  en cuanto se vuelva verdadero, este permanece así siempre.
  
  \textbf{Notación:} $\sigma_i \models LP_i$. Indica que el estado local $\sigma_i$
  de $p_i$ satisface el predicado $LP_i$.
  
  \textbf{Def.} Sea $\{p_1, \dotsm, p_n\}$ un sistema distribuido y $LP_1, \dotsm, LP_n$ $n$
  predicados locales, uno por proceso (con su respectivo proceso). Un \textit{estado global
  consistente} $\sum = (\sigma_1, \dotsm, \sigma_n)$ satisface el predicado global
  $LP_1 \land \dotsm \land LP_n$ denotado por
  \[\sum \models \bigwedge_i LP_i\]
  siempre que $\bigwedge_i (\sigma_i \models LP_i)$.
  \setblockstyle{native} % Default behavior, optional line.
  \begin{center}
    \begin{minipage}[b]{0.5\textwidth}
      \begin{exampleblock}{Problema.}
        \justifying
        Detectemos sobre la \textit{historia} del sistema, y sin utilizar controles de mensajes
        adicionales, el primer estado global consistente $\Sigma$ que satisface una conjunción
        de predicados locales estables.
      \end{exampleblock}    
    \end{minipage}
  \end{center}
\end{frame}
