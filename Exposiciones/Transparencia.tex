\documentclass[9pt]{beamer}

%~~~~~~~~~~~~~~~~~~~~~~~~~~~~~~~~~~~~~~~~~~~~~~~~~~~~~~~~~~~~~~~~~~~~~~~~~~~~~~
% Code
\newcommand{\code}[1]{\textcolor{white!25!black}{\texttt{#1}}}
%~~~~~~~~~~~~~~~~~~~~~~~~~~~~~~~~~~~~~~~~~~~~~~~~~~~~~~~~~~~~~~~~~~~~~~~~~~~~~~

%~~~~~~~~~~~~~~~~~~~~~~~~~~~~~~~~~~~~~~~~~~~~~~~~~~~~~~~~~~~~~~~~~~~~~~~~~~~~~~
% Use roboto Font (recommended)
\usepackage[sfdefault]{roboto}
\usepackage[utf8]{inputenc}
\usepackage[T1]{fontenc}
%~~~~~~~~~~~~~~~~~~~~~~~~~~~~~~~~~~~~~~~~~~~~~~~~~~~~~~~~~~~~~~~~~~~~~~~~~~~~~~

%~~~~~~~~~~~~~~~~~~~~~~~~~~~~~~~~~~~~~~~~~~~~~~~~~~~~~~~~~~~~~~~~~~~~~~~~~~~~~~
% Define where theme files are located. ('/styles')
\usepackage{styles/fluxmacros}
\usefolder{styles}
% Use Flux theme v0.1 beta
% Available style: asphalt, blue, red, green, gray 
\usetheme[style=asphalt]{flux}
%~~~~~~~~~~~~~~~~~~~~~~~~~~~~~~~~~~~~~~~~~~~~~~~~~~~~~~~~~~~~~~~~~~~~~~~~~~~~~~

%~~~~~~~~~~~~~~~~~~~~~~~~~~~~~~~~~~~~~~~~~~~~~~~~~~~~~~~~~~~~~~~~~~~~~~~~~~~~~~
% Extra packages for the demo:
\usepackage{booktabs}
\usepackage{colortbl}
\usepackage{ragged2e}
\usepackage{schemabloc}
%~~~~~~~~~~~~~~~~~~~~~~~~~~~~~~~~~~~~~~~~~~~~~~~~~~~~~~~~~~~~~~~~~~~~~~~~~~~~~~

%~~~~~~~~~~~~~~~~~~~~~~~~~~~~~~~~~~~~~~~~~~~~~~~~~~~~~~~~~~~~~~~~~~~~~~~~~~~~~~
% Informations
\title{Computación distribuida.}
\subtitle{Aplicaciones de Relojes Vectoriales.}
\author{Integrantes:\\
        Aguilera Moreno Adrian\\
        Torres Valencia Kevin Jair\\
        Pérez Romero Natalia Abigail}
\institute{Facultad de Ciencias, UNAM}
\date{\today}
\titlegraphic{Imagenes/im1.png}
%~~~~~~~~~~~~~~~~~~~~~~~~~~~~~~~~~~~~~~~~~~~~~~~~~~~~~~~~~~~~~~~~~~~~~~~~~~~~~~

\begin{document}

% Generate title page
\titlepage

\begin{frame}
 \frametitle{Tabla de contenido.}
 \tableofcontents
\end{frame}

\section{Introducción.}
%%%%%%%%%%%%%%%%%%% Pruebas
\begin{frame}{Introducción}{Historia...}
	\justifying
        Los \underline{relojes vectoriales} son un tipo de
        reloj lógico propuesto de manera independiente por
        \textit{Colin J. Fidge} y \textit{Friedemann Mattern}
        en 1988.
        
        \begin{center}
          \includegraphics[height = 2cm]{./Imagenes/FidgeAndMattern.png}
        \end{center}
        
        Esta técnica consiste en un mapeo entre eventos en una
        historia distribuida y vectores enteros.
\end{frame}


\subsection{Tiempo Vectorial.}
%%%%%%%%%%%%%%%%%%% Especificación:
\begin{frame}[fragile]{Definiciones:}{Vector Tiempo.}
  \justifying
  \textbf{Sistema Vectorial de Relojes.} Es un mecanismo
  capaz de caracterizar estados locales (en adelante, eventos)
  en un sistema distribuido, asociando un valor vectorial
  a cada estado.\\[0.3cm]
  %  Esto nos permite saber que relación hay entre estos estados.
  
  \textbf{Tiempo Vectorial.} Es la noción de tiempo capturada
  por los relojes vectoriales.\\[0.3cm]

  \textbf{Caracterización Formal del Tiempo Vectorial.} Sea
  \code{date(e)} la caracterización asociada a un evento \code{e},
  de tal manera que se cumple:
  \begin{enumerate}
  \item $\forall_{e_1, e_2}:\left(e_1 \rightarrow e_2\right)
    \Leftrightarrow \code{date($e_1$)} < \code{date($e_2$)}$.
  \item $\forall_{e_1, e_2}: \left(e_1\ ||\ e_2\right) \Leftrightarrow
    \code{date($e_1$)}\ ||\ \code{date($e_2$)}$.
  \end{enumerate}
  \begin{figure}
    \includegraphics[height = 2.5cm]{./Imagenes/RelojVectorialSimple.png}
    \caption{Reloj Vectorial con tiempos locales.}
  \end{figure}
\end{frame}


\subsection{Relojes Vectoriales.}
%%%%%%%%%%%%%%%%%%% Especificación:
\begin{frame}[fragile]{Definiciones:}{Reloj Vectorial.}
    \justifying
    \textbf{Reloj Vectorial.} La implementación del tiempo
    vectorial requiere que cada proceso, en el sistema,
    mantenga un vector de enteros positivos $Vc_i[1, \dotsm, n]$
    con valores inicialmente $[0, \dotsm, 0]$. Este vector
    debe cumplir con
    \begin{enumerate}
    \item $Vc_i[i]$ cuenta el número de eventos producidos por
      $p_i$.
    \item $Vc_i[j]$, $j \not= i$, nos dice cuántos eventos conoce
      $p_i$ producido por $p_j$.
    \end{enumerate}
    De manera formal, sea $e$ un evento producido por $p_i$,
    tenemos que
    \[Vc_i[k] = \left|\{f | (f \text{ se produjo por } p_k) \land
    (f \rightarrow e)\}\right| + 1(k, i).\]
\end{frame}


\def\beamer@mytheme@style{green}

\subsection{Algoritmo.}
%%%%%%%%%%%%%%%%%%% Especificación:
\begin{frame}{Reloj Vectorial:}{Algoritmo.}
  \justifying
  \textbf{Una primera aproximación.} Inicialmente
  todos los procesos disponen de un vector con
  entradas igual al número total de procesos, este
  vector debe estar inicializado en $0$ para cada
  entrada. A continuación se describe el algoritmo:
  \begin{itemize}
  \item[$\blacktriangleright$] Si $p_i$ produce un evento, entonces:
    \begin{enumerate}
    \item[(1)] $Vc_i[i] \leftarrow Vc_i[i] + 1$;
    \item[(2)] Produce un evento $e$ caracterizado por $Vc_i[1, \dotsm, n]$.
    \end{enumerate}
  \item[$\blacktriangleright$] Cuando $p_i$ envia un mensaje a $p_j$, entonces:
    \begin{enumerate}
    \item[(3)] $Vc_i[i] \leftarrow Vc_i[i] + 1$;
    \item[(4)] \code{send(\textlangle msj, $Vc_i$[1, $\dotsm$, n] \textrangle)} a $p_j$.
    \end{enumerate}
  \item[$\blacktriangleright$] Cuando $p_j$ recibe un mensaje, entonces:
    \begin{enumerate}
    \item[(3)] $Vc_j[j] \leftarrow Vc_j[j] + 1$;
    \item[(4)] $Vc_j[1, \dotsm, n] \leftarrow \forall_{k \in [1, \dotsm, n]}
      \code{max}(Vc_i[k], Vc_j[k])$.
    \end{enumerate}
  \end{itemize}
\end{frame}
 % Algoritmo.
%%%%%%%%%%%%%%%%%%% Especificación:
\begin{frame}{Algoritmo:}{Propagación del tiempo vectorial.}
  \justifying
  \textbf{Notación:} Para, cualesquiera, dos vectores $Vc_1$
  y $Vc_2$ del tamaño $n$. Tenemos que
  \begin{itemize}
  \item[$\blacktriangleright$] $Vc_1 \leq Vc2 =_{def.}
    \left(\forall_{k \in \{1, \dotsm, n\}} : Vc_1[k] \leq Vc_2[k]\right)$;
  \item[$\blacktriangleright$] $Vc_1 < Vc2 =_{def.}
    \left(Vc_1[k] \leq Vc_2[k]\right) \land  \left(Vc_1[k] \not= Vc_2[k]\right)$;
  \item[$\blacktriangleright$] $Vc_1 || Vc_2 =_{def} \neg (Vc_1 \leq Vc_2) \land
    \neg (Vc_2 \leq Vc_1)$.
  \end{itemize}
  \begin{figure}
    \includegraphics[height = 4.5cm]{./Imagenes/RelojVectorialCompuesto.png}
    \caption{Ejemplo de propagación en un reloj Vectorial.}
  \end{figure}
\end{frame}
 % Propagación del tiempo vectorial.
%%%%%%%%%%%%%%%%%%% Especificación:
\begin{frame}[fragile]{Algoritmo:}{Propiedades.}
  \justifying
  \textbf{Def.} Sea $e.Vc$ el vector asociado al evento $e$.\\[0.3cm]
  
  \textbf{Teo 1.} Por el algoritmo mencionado tenemos que, para cualesquiera
  $e_1$ y $e_2$ distintos tenemos que
  \begin{enumerate}
  \item[$a$)] $\left(e_1 \rightarrow e_2\right) \Leftrightarrow \left(e_1.Vc < e_2.Vc\right)$;
  \item[$b$)] $\left(e_1 || e_2\right) \Leftrightarrow \left(e_1.Vc || e_2.Vc\right)$.
  \end{enumerate}
  
  \textbf{Cor 1.} Dadas dos caracterizaciones a eventos (fechas), determinar si estos
  eventos están relacionados o no, puede requerir hasta $n$ comparaciones de enteros.
  \\[0.3cm]
  
  \textbf{Teo 2.} Sean dos eventos $e_1$ y $e_2$ con tuplas $\langle e_1.Vc, i \rangle$
  y $\langle e_2.Vc, j\rangle$ de manera respectiva y $i \not= j$. Entonces
  \begin{enumerate}
  \item[$a$)] $\left(e_1 \rightarrow e_2\right) \Leftrightarrow
    \left(e_1.Vc[i] \leq e_2.Vc[i]\right)$;
  \item[$b$)] $\left(e_1 || e_2\right) \Leftrightarrow
    \left((e_1.Vc[i] > e_2.Vc[i]) \land (e_2.Vc[j] > e_1.Vc[j])\right)$.
  \end{enumerate}
\end{frame}
 % Propiedades. 
%%%%%%%%%%%%%%%%%%% Especificación:
\begin{frame}{Algoritmo:}{Reducción de costo en la comparación de dos vectores.}
  \justifying
  \textbf{Mejora en la complejidad en tiempo.} Hasta el momento la complejidad
  en tiempo para combinar $2$ eventos nos toma $\mathcal{O}(n)$, con $n$ el
  número de procesos en el sistema.\\[0.3cm]

  Por el \textit{Teo. 2}, sabemos que basta con verificar dos entradas para
  saber como es un evento respecto al otro. Así, basta comparar dos entradas
  para combinar la caracterización de $2$ eventos esto nos toma $\mathcal{O}(1)$.
\end{frame}
 % Reducción de costo en la comparación de dos vectores.
%%%%%%%%%%%%%%%%%%% Especificación:
\begin{frame}[fragile]{Algoritmo:}{Relación del tiempo vectorial y estados globales.}
  \justifying
  Consideremos 
\end{frame}
 % Relación del tiempo vectorial y estados globales.

\subsection{Desventajas.}
%%%%%%%%%%%%%%%%%%% Especificación:
\begin{frame}[fragile]{Relojes Vectoriales:}{Desventajas.}
  Esta mejora a los relojes lógicos de Lamport tiene un problema
  de implementación, que en un momento será más evidente. 

  \setblockstyle{native} % Default behavior, optional line.
  \begin{center}
    \begin{minipage}[b]{0.5\textwidth}
      \begin{exampleblock}{Desventaja}
        Esta desventaja es que cada proceso tiene que cargar con espacio
        igual al número de procesos en el sistema y cada intercambio
        entre eventos es de este tamaño.
      \end{exampleblock}    
    \end{minipage}
  \end{center}
\end{frame}
  % Desventajas.

% The [plain] causes the headlines, footlines, and sidebars 
% to be suppressed. Useful for showing large pictures

% TODO. Sin implementar.
\section{Aplicaciones.}
\subsection{El caso DynamoDB.}
\input{./Problema09}
\subsection{Un problema de conjuntos.}
%%%%%%%%%%%%%%%%%%% Especificación:
\begin{frame}{En proceso}{...}
  
\end{frame}

\subsection{Determinando Propiedades Globales.}
%%%%%%%%%%%%%%%%%%% Especificación:
\begin{frame}[plain]
	\begin{center}
	  This is a plain frame.\\
	  Use it to display full page images.
	  \end{center}
\end{frame}

\begin{frame}[allowframebreaks]{References}

  \nocite{*}
  \bibliography{demo}
  \bibliographystyle{abbrv}

\end{frame}

\subsection{Implementación en sistemas dinámicos.}
\subsection{Detección de una conjunción de predicados locales estables.}

\section{Relojes de Bloom.}
\subsection{Introducción.}
\subsection{Filtro Bloom.}
\subsection{Relojes de bloom.}
\subsection{Aplicaciones y comparación con los relojes vectoriales.}
%%%%%%%%%%%%%%%%%%% Especificación:
\begin{frame}
  
\end{frame}

\end{document}
