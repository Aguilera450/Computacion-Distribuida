%%%%%%%%%%%%%%%% Relojes Bloom.
\begin{frame}[fragile]{Relojes Bloom:}{Filtro Bloom. Mejoras.}
  \justifying
  ¿Qué podemos hacer? ...\newline
  Podemos controlar la tasa de falsos positivos ajustando:
  \begin{enumerate}
  \item El tamaño del \textit{bloom-filtro} ($m$).
  \item El número de elementos insertados en el filtro ($n$).
  \item El número de \textit{hash-funciones} ($k$) utilizadas en la codificación.
  \end{enumerate}
  y por medio de la ecuación:
  \[\left(1 - \left(1 - \frac{1}{m}\right)^{kn}\right)^k\]
  \textbf{Idea Intuitiva:}
  \begin{itemize}
  \item $1 - \frac{1}{m}$ es la probabilidad de tener un \code{bit} en
    el filtro que no este establecido como $1$ por alguna determinada
    función.
  \item $\left(1 - \frac{1}{m}\right)^k$ es la probabilidad de que un bit
    en particular no se establezca en uno, dado que $k$ hashes pueden
    señalarlo potencialmente.
  \item $\left(1 - \frac{1}{m}\right)^{kn}$ es la probabilidad de que un
    bit en particular siga siendo cero después de $n$ elementos.
  \item $\left(1 - \left(1 - \frac{1}{m}\right)^{kn}\right)^k$ es la probabilidad
    de que $k$ índices sean $1$ después de insertar n elementos, también
    conocido como tasa de falsos positivos.
  \end{itemize}
\end{frame}
