%%%%%%%%%%%%%%%%%%% Especificación:
\begin{frame}[fragile]{Algoritmo:}{Propiedades.}
  \justifying
  \textbf{Def.} Sea $e.Vc$ el vector asociado al evento $e$.\\[0.3cm]
  
  \textbf{Teo 1.} Por el algoritmo mencionado tenemos que, para cualesquiera
  $e_1$ y $e_2$ distintos tenemos que
  \begin{enumerate}
  \item[$a$)] $\left(e_1 \rightarrow e_2\right) \Leftrightarrow \left(e_1.Vc < e_2.Vc\right)$;
  \item[$b$)] $\left(e_1 || e_2\right) \Leftrightarrow \left(e_1.Vc || e_2.Vc\right)$.
  \end{enumerate}
  
  \textbf{Cor 1.} Dadas dos caracterizaciones a eventos (fechas), determinar si estos
  eventos están relacionados o no, puede requerir hasta $n$ comparaciones de enteros.
  \\[0.3cm]
  
  \textbf{Teo 2.} Sean dos eventos $e_1$ y $e_2$ con tuplas $\langle e_1.Vc, i \rangle$
  y $\langle e_2.Vc, j\rangle$ de manera respectiva y $i \not= j$. Entonces
  \begin{enumerate}
  \item[$a$)] $\left(e_1 \rightarrow e_2\right) \Leftrightarrow
    \left(e_1.Vc[i] \leq e_2.Vc[i]\right)$;
  \item[$b$)] $\left(e_1 || e_2\right) \Leftrightarrow
    \left((e_1.Vc[i] > e_2.Vc[i]) \land (e_2.Vc[j] > e_1.Vc[j])\right)$.
  \end{enumerate}
\end{frame}
