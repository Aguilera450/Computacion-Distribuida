\documentclass{article}
\usepackage[utf8]{inputenc}
\usepackage[spanish,mexico]{babel}
\usepackage[margin=2.3cm]{geometry}
\usepackage{graphicx}
\usepackage[export]{adjustbox}
\usepackage{caption}
\usepackage{subcaption}
\usepackage{fancyhdr}
\usepackage{lipsum}
\usepackage{float}
\usepackage{amsfonts, amsthm}
\usepackage{upgreek}
\usepackage{physics}
\usepackage{cancel}
\usepackage{amssymb, latexsym, amsmath}
\usepackage{amsthm}

% Algorithm
\usepackage{algorithm}
\usepackage{algpseudocode}
%\usepackage[ruled,lined,linesnumbered,commentsnumbered]{algorithm2e}

% Notas al pie.
\renewcommand{\thefootnote}{\Roman{footnote}}

% Evitar sangrías
\setlength{\parindent}{0cm}

% Letter colors.
\usepackage[usenames,dvipsnames,svgnames,table]{xcolor}

% Código
\newcommand{\code}[1]{\textcolor{white!25!black}{\texttt{#1}}}
\usepackage{listings}

% Def. Dr. César.
\usepackage{tikz}
\usetikzlibrary{shapes,calc}
\tikzstyle{edge}=[shorten <=2pt, shorten >=2pt, >=stealth, line width=1.1pt]
\tikzstyle{blueE}=[shorten <=2pt, shorten >=2pt, >=stealth, line width=1.5pt, blue]
\tikzstyle{blackV}=[circle, fill=black, minimum size=6pt, inner sep=0pt, outer sep=0pt]
\tikzstyle{blueV}=[circle, fill=blue, draw, minimum size=6pt, line width=0.75pt, inner sep=0pt, outer sep=0pt]
\tikzstyle{redV}=[circle, fill=red, draw, minimum size=6pt, line width=0.75pt, inner sep=0pt, outer sep=0pt]
\tikzstyle{redSV}=[semicircle, fill=red, minimum size=3pt, inner sep=0pt, outer sep=0pt, rotate=225]
\tikzstyle{blueSV}=[semicircle, fill=blue, minimum size=3pt, inner sep=0pt, outer sep=0pt, rotate=225]
\tikzstyle{blackSV}=[semicircle, fill=black, minimum size=3pt, inner sep=0pt, outer sep=0pt, rotate=225]
\tikzstyle{vertex}=[circle, draw, minimum size=6pt, line width=0.75pt, inner sep=0pt, outer sep=0pt]

% Proof
\renewcommand*{\proofname}{\textbf{Soluci\'on:}}

% Margins
%\addtolength{\textheight}{3cm}

\usepackage{amsmath}
\usepackage{amsmath}
\usepackage{amssymb}
\usepackage{amsthm}
\usepackage{url}
\usepackage{hyperref}

%%Tablas con color
\usepackage{xcolor, colortbl}
\usepackage{array, multirow, multicol}
\cellcolor[modelo color]{color}

\pagestyle{fancyplain}

\title{Facultad de Ciencias, UNAM}
\date{\today}

\hypersetup{
  colorlinks=true,
  linkcolor=blue,
  filecolor=magenta,      
  urlcolor=cyan,
  pdftitle={Overleaf Example},
  pdfpagemode=FullScreen,
}

\begin{document}
\fancyhead[r]{ Computación Distribuida 2023-1}
\thispagestyle{empty}

\begin{figure}[ht]
  \minipage{0.76\textwidth}
  \includegraphics[width=4cm]{Logo_UNAM.png}
  \label{EscudoUNAM}
  \endminipage
  \minipage{0.32\textwidth}
  \includegraphics[height = 4.9cm ,width=4cm]{Logo_FC.png}
  \label{EscudoFC}
  \endminipage
\end{figure}

\begin{center}
  \vspace{0.8cm}
  \LARGE
  UNIVERSIDAD NACIONAL AUTÓNOMA DE MÉXICO 
  
  \vspace{0.7cm}
  \LARGE
  FACULTAD DE CIENCIAS
  
  \vspace{0.8 cm}	
  \Large
  \textbf{Tarea 2}

  \vspace{0.8 cm}
  \normalsize	
  INTEGRANTES \\
  \vspace{.2cm}
  \large
  \textbf{Torres Valencia Kevin Jair - \texttt{318331818}}\\
  \textbf{Aguilera Moreno Adrián - \texttt{421005200}}\\
  \textbf{Natalia Abigail Pérez Romero  - \texttt{318144265}}\\
  %\textbf{Nombre - \texttt{Número de cuenta}}
  
  \vspace{1 cm}
  \normalsize	
  PROFESOR \\
  \vspace{.2cm}
  \large
  \textbf{Miguel Ángel Piña Avelino}
  
  \vspace{1 cm}
  AYUDANTE \\
  \vspace{.2cm}
  \large
  \textbf{Pablo Gerardo González López}
  \vspace{1.3cm}
  
  \normalsize	
  ASIGNATURA \\
  \vspace{.2cm}
  \large
  \textbf{Computación Distribuida}
  
  \vspace{1 cm}
  \today
\end{center}

%%%%%%%%%%% Aquí va la solución al problema 1.
\newpage
\textbf{\textcolor{MidnightBlue}{1.}}
Investiga y explica brevemente el concepto de time-to-live (TTL) usado en redes de
computadoras, y úsalo para modificar el algoritmo de flooding visto en clase, de modo que un
líder comunique un mensaje m a los procesos a distancia a lo más d del líder (m y d son entradas
del algoritmo); todos los procesos a distancia mayor no deberán recibir m. Da un breve argumento
que demuestre que tu algoritmo es correcto, y también haz un análisis de tiempo y número de
mensajes.
%%%%%%%%%%%%%%%%%%% Especificación:
\begin{frame}[fragile]{Flux}{fonts}
 Flux recommends the use of Roboto or Overpass font and a font size of 9pt.\\[0.2cm]
 \begin{center}
 	\verb+\documentclass[9pt]{beamer}+\\
	\verb+\usepackage[sfdefault]{roboto}+
 \end{center}
 
  Flux also implements default typographies.

	\begin{itemize}
		\item Regular
		\item \alert{Alert}
		\item \example{Example}
		\item \textit{Italic}
		\item \textbf{Bold}
	\end{itemize}
	
\end{frame}

%%%%%%%%%%% Aquí va la solución al problema 3.
\textbf{\textcolor{MidnightBlue}{3.}} Considera el algoritmo de consenso con
fallas de tipo paro visto en clase. Suponga que en lugar de ejecutar $f + 1$
rondas, el algoritmo sólo ejecuta $f$, con la misma regla de decisión. Describa
una ejecución particular en la que las propiedades de validez y acuerdo sean violadas.
\newline

\begin{center}
    \includegraphics[scale=0.5]{Consenso.jpeg}
    \end{center}

$\rhd$ Para esta ejecución basta observar que puede pasar en cada
ronda si resultan $2$ fallos de tipo paro en un sistema distribuido
con $4$ procesos. Esto es
\begin{itemize}
\item $\mathbf{R_1}.$ Para esta ronda todos los procesos mandan su \code{ID}
      exceptuando al proceso $p_1$ que logra mandar su \code{ID} al proceso
      $p_2$ y después tiene un problema tipo fallo y no completa su envio a
      los demás procesos.
\item $\mathbf{R_2.}$ En la $2^{\text{da}}$ ronda el proceso $p_2$ manda su
      \code{ID} a $p_4$ pero no logra enviarselo al proceso $p_3$. Entonces
      $p_2$ es nuestro segundo proceso con fallo de tipo paro.
\end{itemize}
El algoritmo solo nos permite realizar $2$ rondas, pues $f = 2$. Así,
la vista de $p_3$ es $\{2, 3, 4\}$ y su propuesta es $prop = 2$. Mientras
que $p_4$ contiene en su vista a $\{1, 2, 3, 4\}$ y su propuesta es
$prop = 1$. Como se puede observar, los procesos restantes en el sistema
no han llegado a un consenso en $f$ rondas.
\hfill $\lhd$

\newpage
\textbf{\textcolor{MidnightBlue}{4.}}
Decimos que un canal de comunicación entre dos procesos $p_i$ y $p_j$ es \code{FIFO} si los
mensajes se reciben en el mismo orden en el que se envían. Entonces, si el canal \code{NO}
es \code{FIFO}, puede ser que $p_i$ envía primero $M_1$ y después $M_2$, pero $p_j$
reciba primero $M_2$ y después $M_1$.\\

Dados dos procesos que comparten un canal $C$ que no es \code{FIFO}, da un algoritmo que
implemente un canal \code{FIFO} sobre $C$. Tu algoritmo debe tener dos secciones: una sección
{\bf send}, que recibe como entrada un mensaje $M$ a enviarse (el cuál se envía finalmente
por $C$), y una sección {\bf receive}, que recibe un mensaje $M$ de $C$ y lo entrega. De esta
forma, otro algoritmo que esté diseñado para canales \code{FIFO} puede usar tu algoritmo para
enviar y recibir mensajes. Este esquema se puede observar en la figura 2. Argumenta que tu
algoritmo es correcto. Tip: piensa en \code{timestamps} y considera que un mensaje que se
recibe de $C$ no tiene que entregarse inmediatamente.

\begin{center}
    \includegraphics[scale=0.5]{Grapho1.png}
    \end{center}

%%%%%%%%%%% Aquí va la solución al problema 5.
\newpage
\textbf{\textcolor{MidnightBlue}{5.}} El algoritmo puede mejorar su complejidad de tiempo si se ejecutan de forma paralela
los dos algoritmos anteriores, es decir, si se ejecuta la elección de líder y la construcción del árbol BFS. Da un algoritmo distribuido que realice esto y muestra que es correcto. Adicionalmente, compara la complejidad de tiempo respecto al algoritmo anterior.

\begin{algorithm}
    \caption{arbolGenerador (ID,total)}\label{alg:cap}
    \begin{algorithmic}[1]
        \State $PADRE = \bot, \ \ HIJOS = \emptyset, \ \ OTROS = \emptyset$
        \State Si no he recibido algún mensaje
        \If{$PADRE == \bot$}
            \State $LIDER = ID$ 
            \State $send(<BFS,ID,LIDER>)$ a todos mis vecinos
            \State $PADRE = ID$
        \EndIf \\


        \State Al recibir $<BFS,ID, LIDER>$ desde el vecino $p_j$

        \If{$PADRE == \bot$}
            \State $PADRE = j$
            \State $mensajes=\{<L_1>,<L_2>,\dots,<L_d>\} \cup LIDER$
            \State $LIDER = max(mensajes)$
            
            \State $send(<parent,LIDER>)$ a $p_j$
            \State $send(<BFS,ID, LIDER>)$ a todos excepto $p_j$
        \Else
            \State $send(<already>)$ a $p_j$
        \EndIf\\


        \State Al recibir $<parent,LIDER>$ dede el vecino $p_j$
        \State $HIJOS = HIJOS \cup \{p_j\}$
        
        \State $mensajes=\{<L_1>,<L_2>,\dots,<L_d>\} \cup LIDER$
        \State $LIDER = max(mensajes)$
        \State $send(<LIDER>)$ a todos en HIJOS y a PADRE\\

        \State Al recibir $<already>$ desde $p_j$
        \State $OTROS = OTROS \cup \{p_j\}$\\


        \State Al recibir $<LIDER>$ desde $p_j$
        
        \If{$LIDER == <LIDER> \ \ and \ \ HIJOS \cup OTROS == VECINOS-PADRE$}
            \State \textbf{Terminar}
        \Else
            \State $LIDER = <LIDER>$
            \State $send(<LIDER>)$ a HIJOS y PADRE
        \EndIf

    \end{algorithmic}
\end{algorithm}    

Podemos describir la ejecución del algoritmo como sigue:
Al principio de la ejecución cualquier nodo que se tome al principio será el lider y la raíz del árbol BFS. Luego le envia $<BFS,ID,LIDER>$ a todos sus vecinos, los cuales si no tienen PADRE asigna $p_j$ como su padre y determinan el LIDER (el maximo de los mensajes $<LIDER>$ que ha recibido), lo asigna como su LIDER, le envia a $p_j$ un mensaje que lo asigna como su padre y le dice el nuevo LIDER, cuando el PADRE recibe el mensaje añade el nodo al conjunto de sus hijos, calcula nuevamente el LIDER y se lo envia a sus HIJOS y a su PADRE. Los cuales registran su nuevo LIDER si este es diferente al nodo que tenian almacenado. Si el nodo ya tiene PADRE entonces envian el mensaje $<already>$ al nodo $p_j$, el cual inserta este al conjunto de sus hijos.

El último segmento lineas 29-35 termina cuando el nodo ya visito a todos sus vecinos (o no tiene hijos) y el LIDER que recibe es igual al que tiene.

\textbf{Afirmación}: El algoritmo {\tt arbolGenerador} es correcto, es decir, cumple con las propiedades de validez y acuerdo. Además construye un árbol con raiz.

\textbf{Acuerdo.} Todos los procesos acuerdan un mismo valor.

Al terminar cualesquiera 2 procesos $p_i$ y $p_j$ con variables $LIDER_i$ y $LIDER_j$, se cumple que $LIDER_i=LIDER_j$ Observemos que para todo tiempo $d>0$, todos los procesos que están a distancia a lo más $d$ del proceso con el ID máximo, tiene ese ID en la variable LIDER.

\textbf{Por inducción sobre d:}

\textbf{Caso base:} $d=0$. Es claro que $LIDER=ID$ para el proceso con el ID máximo.

\textbf{Hipótesis de inducción:} Para todo proceso a distancia $d-1$ del proceso con ID máximo tiene dicho $ID$ en su variable LIDER.

\textbf{Paso inductivo:}
Consideremos un proceso $p_i$ a distancia $d$ del proceso con el ID máximo. A partir de la H.I, sabemos que existe un proceso $p_j$ a distancia $d-1$ del proceso con ID máximo con $LIDER = ID.p_j$ vecino de $P_i$

Por la ejecución del algoritmo sabemos $LIDER = ID.p_j$ cuando al recibir $<LIDER>$ sucede que 
$$LIDER == <LIDER> \ \ and \ \ HIJOS \cup OTROS == VECINOS-PADRE$$ 
y tenemos dos casos $p_j$ recibio el mensaje de su padre o de sus hijos, si lo recibio de su padre en este caso $p_i$ tiene el $LIDER=$ID máximo. Si lo recibio de sus hijos entonces tenemos otros dos casos el $ID$ de $p_j$ era el ID máximo o no lo era; en caso de que no lo fuera envia el nuevo ID máximo a su padre y a sus hijos, quienes caen nuevamente en este caso. Al final de la ejecución $LIDER == <LIDER>$ es decir el ID máximo que recibe del PADRE y de los HIJOS es el mismo.

\textbf{Validez.} Al terminar el algoritmo todos los procesos tienen como LIDER un ID que fue entrada de algún proceso.

Al terminar todo proceso tiene como LIDER un ID que fue entrada de algún proceso. Esto es fácil de observar, ya que $LIDER=max(mensajes)$ y esto es construido a partir de las propuestas de los vecinos.\\


Podemos garantizar que la gráfica que regresa {\tt arbolGenerador} es un árbol con raíz por los argumentos que se usan en BFS:
1. Todo nodo es alcanzable desde la raíz si el sistema es conexo. Dado que a todo nodo se le asigna un padre es una contradicción suponer que no es alcanzable.
2. No tiene ciclo. Si existiera un ciclo entonces un proceso tendría dos padres pero esto es una contradición porque un nodo solo tiene un padre por la linea 11.


Compara la complejidad de tiempo respecto al algoritmo anterior.

En este caso el algortimo hace a BFS al mismo tiempo que elige un lider por lo que tarda la máxima distancia entre el LIDER y cualquier vértice de la gráfica, es decir, su complejidad de tiempo es $O(diam(G))$
\end{document}
