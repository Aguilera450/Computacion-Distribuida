%%%%%%%%%%% Aquí va la solución al problema 5.
\newpage
\textbf{\textcolor{MidnightBlue}{5.}} Un toro $n \times m$ es una versión
dos dimensional de un anillo, donde un nodo en la posición $(i, j)$ tiene
un vecino hacia el norte en $(i, j - 1)$, al este en $(i + 1, j)$, al sur
en $(i, j + 1)$ y al oeste en $(i - 1, j)$. Esos valores se calculan módulo
$n$ para la primera coordenada y módulo $m$ para la segunda; de este modo
$(0, 0)$ tiene vecinos $(0, m - 1)$, $(1, 0)$, $(0, 1)$ y $(n - 1, 0)$.
Supongamos que tenemos una red síncrona de paso de mensajes en forma de un
toro $n \times m$, consistente de procesos anónimos idénticos, los cuáles
no conocen n, m o sus propias coordenadas, pero tienen sentido de la dirección
(es decir, puede decir cual de sus vecinos está al norte, este, etc.).
\textbf{Pruebe o refute}: Bajo estas condiciones, ¿existe un algoritmo
determinista que calcule cuando $n > m$?
