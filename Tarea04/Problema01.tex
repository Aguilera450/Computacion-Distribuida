%%%%%%%%%%% Aquí va la solución al problema 1.
\newpage
\textbf{\textcolor{MidnightBlue}{1.}}
Describe un algoritmo distribuido basado en $DFS$ que cuente el número de procesos
en un sistema distribuido cuya gráfica $G$ es arbitraria. Al terminar de contar, debe informar a todos
los procesos el resultado del conteo. Muestra que es correcto.


\begin{algorithm}
\caption{contarProcesosDFS(ID,soyLider)}
\begin{algorithmic}[1]
\State $Padre = \bot$
\State $Hijos = \emptyset$
\State numProcesos = 0
\State SinExplorar = todos los vecinos

\State Si no he recibido algún mensaje

\If{soyLider and Padre $== \bot$}
    \State Padre = ID
    \State numProcesos = 1
    \State explore(numProcesos)
\EndIf

\State Al recibir $<numP>$ desde el vecino $p_j$:

\If{Padre $== \bot$}
    \State Padre = j
    \State numProcesos = numP + 1 
    \State elimina $p_j$ de SinExplorar
    \State explore(numProcesos)
\Else
    \If{$numProcesos < numP$}
        \State numProcesos = numP
    \EndIf

    \State send($<already>$) a $p_j$:
    \State elimina $p_j$ de SinExplorar
\EndIf

\State Al recibir $<already>$ desde $p_j$
    \State explore(numProcesos)

\State Al recibir $<parent,numP>$
    \If{$numProcesos < numP$}
        \State numProcesos = numP
    \EndIf

    \State Hijos $\cup {p_j}$
    \State explore(numProcesos)


\Procedure{explore}{numP}
    \If{SinExplorar $\neq \emptyset$}
        \State elegir $p_k$ en SinExplorar
        \State eliminar $p_k$ de SinExplorar
        \State send $(<numP>)$ a $p_k$
    \Else
        \If{Padre $\neq$ ID}
            \State send($<parent, numP>$) a Padre
        \EndIf
        
        \If{Padre $==$ ID and numProcesos $<$ numP}
            \State numProcesos = numP
        \EndIf
    \EndIf
\EndProcedure
\end{algorithmic}
\end{algorithm}

\textbf{Caso base.} 

Sea $G$ una gráfica tal que $V_G=\{p_1\}$, por la linea 8 $contarProcesosDFS()= 1 $ lo cual es correcto.

\textbf{Hipótesis de inducción}

Para cualquier gráfica $G$ con $n$ vértices {\tt contarProcesosDFS()} cuenta la cantidad procesos de $G$

Al inicio de la ejecución si no

\textbf{Paso inductivo}

Por demostrar que dada una gráfica $G$ con $n+1$ vértices {\tt asignarEtiquetasDFS()} asigna etiquetas únicas en el rango $[1,\dots, n+1]$.