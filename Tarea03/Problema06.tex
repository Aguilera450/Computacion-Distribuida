%%%%%%%%%%% Aquí va la solución al problema 6.
\newpage
\textbf{\textcolor{MidnightBlue}{6.}}
Prueba la siguiente afirmación:

El algoritmo BFS construye un árbol enraízado
sobre un sistema distribuido con $m$ aristas
y diámetro D, con complejidad de mensajes
$O(m)$ y complejidad de tiempo $O(D)$. \newline

$\rhd$ Para este problema damos por hecho que
       \code{BFS} construye un árbol enraízado
       de diámetro $D$ (esto por la afirmación 9 y 11).

       Ahora, analicemos las complejidades de este algoritmo,
       para esto observemos dos posibles casos:
       \begin{enumerate}
       \item La complejidad de tiempo es $O(D)$.\newline 
       
       Sea $G$ una gráfica conexa de diametro $D$ a la cuál
       se le aplica \code{BFS} y nos genera $T$ con algún
       vértice distinguido $v$ como líder.
       
       Por las líneas 7 - 9, el líder $v$ se convierte en un
       nodo enraízado para $T$. Ahora, ¿cuál es el número de
       rondas antes de que nuestro algoritmo termine para todos
       los procesos en $G$?, de acuerdo al algoritmo los
       descendientes de $v$ en $T$ serán el resto de los nodos.
       
       El algoritmo, por proceso $p_i$, termina cuando
       \code{Hijos $\cup$ Otros} contiene a todos los vecinos
       del respectivo $p_i$, a excepción de su padre (líneas
       21 y 26). Así, una vez que \code{BFS} inicia en $v$
       los $p_i$ restantes no hacen nada (\code{BFS} ha iniciado
       pero no tiene instrucciones de distribuirse aún), en cuanto
       los vecinos de $p_i$ empiezan a realizar sus tareas,
       empiezan a propagar su $ID$ para convertirse en padre, si
       es que se cumple la condición necesaria.
       
       Pensemos en la trayectoria $t$  más larga que inicia en $v$ y
       termina en el último proceso, $p$, en terminar. Esta
       trayectoria tiene longitud equivalente al diámetro de $G$,
       pues sino fuese así pasaría que
       \begin{enumerate}
       \item La longitud de $t$ es menor que \code{diam(G)}. En este
       caso, existe (por definición de diámetro) algún proceso que
       al conectarse con $t$ por alguna arista $e$ y que tiene como
       vecino a $p$, permite que $t$ siga siendo trayectoria y de hecho
       $t-e$ tiene longitud $1$ mayor que $t$, lo que contradice que
       $t$ fuera la trayectoria más larga.
       
       \item La longitud de $t$ es mayor que \code{diam(G)}. En este
       caso, $t$ debe contener al menos un ciclo por lo que $t$ no
       sería trayectoria y esto es contradictorio.
       \end{enumerate}
       En conclusión $t$ es  de longitud igual que el diámetro de
       $G$. En el momento en que $p$ recibe \code{parent} desde
       alguno de sus puertos, ya el resto nodos han recibido
       \code{parent} o lo están recibiendo de manera simultánea.
       Esto pasa en la ronda $D -1$, pues es en la ronda $0$ donde
       inicia a realizar sus tareas al menos un proceso ($v$).
       En la siguiente ronda envia realiza sus tareas y recibe
       sus respectivos mensajes. Si no terninara en esta ronda,
       entonces hay un proceso que no tiene padre y $p$ no es
       último!!, esto es contradictorio. Como $p$ termina en la
       ronda siguiente a la que le llega padre, podemos notar
       que este último proceso termina en tiempo
       \[(D - 1) + 1 = D \in O(D)\]
       
       
       
       \item La complejidad de mensajes es $O(m)$. \newline
              
       Sabemos que cada proceso envía sus respectivos mensajes
       a sus vecinos, exceptuando a su padre. Eventualmente
       los procesos ocupan como puerto de salida a todas las
       aristas que conecten a sus vecinos, como la gráfica es
       conexa, entonces al finalizar el algoritmo se utilizan
       todas las aristas.

       En el peor de los casos, un proceso que recibe \code{parent}
       ya tiene asignado a un padre, entonces debe regresarle
       \code{already} por el mismo puerto. La complejidad de mensajes
       será, entonces
       \[m + c \cdot m \in O(m)\]
       donde $c$ nos indica la cantidad de procesos que caén en este
       peor caso, como sabemos que esto no pasa cuando se recibe
       \code{parent}, el número de peores caso no es $m$.
       
       \end{enumerate}

\hfill $\lhd$
