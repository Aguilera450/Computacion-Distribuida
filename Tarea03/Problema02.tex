%%%%%%%%%%% Aquí va la solución al problema 2.
\newpage
\textbf{\textcolor{MidnightBlue}{2.}}
 Considera un sistema distribuido representado como una gráfica de tipo anillo, cuyos
canales son bidireccionales, con n = mk procesos, con m > 1 y k es impar. Los procesos en las
posiciones $0, k, 2k, . . . ,(m-1)k$ son marcados inicialmente como líderes, mientras que procesos
en otras posiciones son seguidores. Todos los procesos tienen un sentido de dirección y pueden
distinguir su vecino izquierdo de su vecino derecho, pero ellos no tienen información alguna
acerca de sus ids.
El algoritmo 1 está destinado a permitir que los líderes recluten seguidores. No es difícil ver que
todo seguidor eventualmente se agrega a sí mismo a un árbol enraízado con padre en algún líder.
Nos gustaría que todos esos árboles tuvieran aproximadamente el mismo número de nodos.

\begin{itemize}
	\item ¿Cuál es el tamaño mínimo y máximo posible de un árbol?
	 \item Dibuja el resultado de una ejecución para el algoritmo con $k = 5$  y $m = 4.$
\end{itemize}