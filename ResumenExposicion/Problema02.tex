%%%%%%%%%%% Aquí va la solución al problema 2.
\newpage
\textbf{\textcolor{MidnightBlue}{2.}}
 Considera el algoritmo de BroadcastTree visto en clase.
 ¿Cuál sería el peor caso en complejidad de tiempo para
 el algoritmo BroadcastTree? Explica detalladamente.\\

 \begin{algorithm}
\caption{BroadcastTree(ID,soyRaiz,M)}\label{alg:cap}
\begin{algorithmic}[1]
\State $PADRE, HIJOS$
\State Ejecutar inicialmente\\
\If{soyRaiz}
    \State $send(<M>)$ a todos en HIJOS
\EndIf
\State Al recibir $<M>$ de Padre
\State $send(<M>)$ a todos en HIJOS
\end{algorithmic}
\end{algorithm}

Supongamos siempre la existencia de un árbol generador $T$.\\
Primero análicemos la complejidad del algoritmo.
Proponemos una gráfica con un vértice $v$, para que el algoritmo pueda terminar,
se necesita que este sea $raiz$. Entonces se tiene que:
$u \{ padre=Null,\; Hijos= \varnothing, \; soyRaiz=True \}$.\\

Como $v$ es el único proceso entonces no envía mensaje a nadie y se cumple que:\\
$Mensajes(BroadcastTree)=|V|-1 \rightarrow 1-1 = 0$\\
$Tiempo(BroadcastTree)= Profundidad(T)=0$\\

Sea $A=\{ u|\; profundidad(u) = max(profundidad(v))\; con \; v\in V\}$, para el
conjunto de procesos $A$ con mayor profundidad.\\
Sea $d=profundidad(T)$. Entonces, para todos los procesos en el nivel $d-1$ son
hoja, por lo tanto esos procesos ya terminaron, pero podría existir un vértice
$v_i$ para cada proceso en $A$, tal que $d(v_i,u_i)=1$ (i.e se repite), entonces la
$profundidad(v_i)\;=\;d-1$. Por lo tanto, en el tiempo $d-1$ se recibió y envió
el mensaje, por lo que en $d$ se termina.\\
En este caso el mejor tiempo de complejidad, es igual a la $profundidad(T)$.
Y para el peor, sería que la $profundidad(T)$ se volviese cuadrática, en caso
de existir hojas que no compartan profundidad con cualquier $u_i\in A$ y no
hayan recibido mensaje.
