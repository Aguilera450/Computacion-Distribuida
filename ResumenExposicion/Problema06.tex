%%%%%%%%%%% Aquí va la solución al problema 6.
\newpage
\textbf{\textcolor{MidnightBlue}{6.}}
Da un algoritmo distribuido para contar el
número de procesos en cada capa de un árbol
enraizado T de forma separada. Al final la
raíz reportará el número de procesos por capa.
Analiza la complejidad de tiempo y la complejidad
de mensajes de tu algoritmo.\\

Necesitariamos de un algoritmo auxilar que nos permite saber que en cada proceso,
su variable asignará la profundidad con su propia profundidad y regresa la
profundidad del Árbol.\\

Se propondría que:\\
El algoritmo se comienza desde la raíz, indicando su propia profundidad y
avisando a todos sus vecinos. Para el caso en el que, la raíz no cuente con
vecinos, su profundidad será cero.\\

Cuando un proceso/vértice recibe un entero, se inicializa su variable de profundidad
conforme al entero que recibio. En el caso de que sea HOJA, entonces termina y
le notificará a su PADRE con su altura, de lo contrario, avisará a todos sus
HIJOS su profundidad.\\

Cuando se reciba $<Acabe, n>$, el proceso indicará, si la $n$ es mayor su
variable $max$, de ocurrir entonces se fijará $max=n$.\\

Si $<Acabe, n>$ es igual al numero de HIJOS, se acabó. De ser el caso de la
raíz, no sregresará su variable $max$, de lo contrario, le avisará a su PADRE
que acabó y su profundidad máxima.\\

Nuestro análisis de tiempo y espacio de complejidad:\\
Su tiempo es la $2\cdot profundidad(T)$, ya que recorre el árbol desde la raíz
hasta las hojas y viceversa. Y se envió $2|E|$ de mensajes

Como ahora cada proceso/vértice podría conocer su profundidad, podriamos
regresar la cardinalidad los HIJOS de cada capa/nivel.\\

\begin{algorithm}
\caption{nivelTree(soyRaiz)}\label{alg:cap}
\begin{algorithmic}[1]
\State profundidadArbol = profundidadT(soyRaiz), nivelT = 0, vecinos = 0
\State $PADRE, HIJOS, profundidad, soyRaiz=soyRaiz, string = ''$
\State Ejecutar inicialmente\\
\If{soyRaiz}
    \State string = '1'
    \If{$|Hijos| == 0$}
    return '1'
    \Else
        \State string = string++ ';' ++ $|HIJOS|$\\
        //Todos los procesos del nivel mandarán la $|HIJOS|$, esto hace la linea 10
        \State $send(<Manda, 1>)$ a todos en HIJOS
    \EndIf
\EndIf\\
\State Al recibir $<Manda, nivel>$ de Padre
\If{profundidad == capa }
    \State  $send(< |HIJOS|, nivel>)$ a Padre
\Else
    \State  $send(<Manda, nivel>)$ a todos en HIJOS
\EndIf\\
\State Al recibir $<HIJOS, nivel>$ de algún proceso en HIJOS
\State vecinos +=, nivelT += HIJOS
\If{$vecinos == |HIJOS|$}
    \If{soyRaiz}
        \State string = string ++ ';' ++HIJOS
        \If{nivel == profundidadArbol-1}
            \State return string
        \Else
            \State $send(<Manda, nivel+1>)$ a HIJOS
        \EndIf
    \Else
        \State $send(<nivelT, nivel>)$ a PADRE
    \EndIf
\EndIf
\end{algorithmic}
\end{algorithm}

\newpage
El algoritmo, recorre desde la raíz a sus hojas (niveles) y viceversa, a
comparación de la mayor profundidad, entonces se tiene que el tiempo toma:
$profundidad(t)*profundidad(T)$

Los mensaje solo se enviarán al nivel de interés: $n =profundidad(T)-1$,
y asi también para el recorrido de abajo hacia arriba. Finalmente nos regresará
un cadena $n_0,n_1,n_2,...,n_i$ donde $n_i$ será el número de vértices por nivel.
