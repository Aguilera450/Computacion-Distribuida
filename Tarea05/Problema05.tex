%%%%%%%%%%% Aquí va la solución al problema 5.
\newpage
\textbf{\textcolor{MidnightBlue}{5.}} El algoritmo de consenso bizantino visto en clase, resuelve el problema del consenso
bizantino para $f < \frac{n}{4}$ procesos. Para valores grandes de $f$, el algoritmo podría fallar por violar una o más de las propiedades de terminación, validez o acuerdo. Para este algoritmo:
\begin{itemize}
    \item ¿Qué tan grande debe ser $f$ para evitar terminación?
    \item ¿Qué tan grande debe ser $f$ para evitar validez?
    \item ¿Qué tan grande debe ser $f$ para evitar acuerdo?
\end{itemize}
Asuma que los procesos conocen la nueva cota $f$, y cualquier umbral en el algoritmo que use $f$, se ajusta para corresponder con esta nueva cota. Argumente detalladamente su respuesta.


