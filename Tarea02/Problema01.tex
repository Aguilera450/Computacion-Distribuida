%%%%%%%%%%% Aquí va la solución al problema 1.
\newpage
\textbf{\textcolor{MidnightBlue}{1.}}
Considera el algoritmo de Flooding visto en clase.
Demuestra el siguiente corolario:                                                        

Todo proceso $p_i$ que ejecuta el algoritmo de
{\tt Flooding}, recibe $M$ en tiempo a lo más el
diámetro $diam(G)$ de la gráfica del sistema
distribuido.
\newline

\textbf{Dem.} Procedamos por inducción en el número
de procesos. Supongamos que nuestros procesos viven
en la gráfica $G$ conexa.
\begin{itemize}
\item Observemos que pasa cuando solo hay un proceso en $G$,
      entonces este es el lider y además tiene el mensaje
      \code{M} desde antes de la primer ronda\footnote{Sabemos
      que el \code{diam(G)} $=$ \code{0}.} (desde la ronda 0).
\item Supongamos que la gráfica $G$, con $k$ procesos, cumple
      con poder inundarse\footnote{Por simplicidad, llamaremos
      inundar a distribuir el mensaje \code{M} por cada proceso
      en $G$ por medio del algoritmo \code{flooding}.} en tiempo
      a lo más \code{diam(G)}.
\item Veamos que pasa cuando tenemos $G'$ conexa donde
      \[V_{G'} = V_G + p_i \hspace*{0.3cm} \text{ y } \hspace*{0.3cm}
      E_{G'} = E_G + \text{ al menos una arista que conecte a $p_i$
      con $G$.}\]
      
      Existen dos casos posibles,
      \begin{enumerate}
      \item \code{diam(G') = diam(G)}. Para este caso tomemos al primer
      vecino $p_j$ de $p_i$ tal que $d(p_j, v)$ sea mínima para todo
      $v \in V$ (en contraste con el resto de vecinos de $p_i$, que no
      necesariamente existen). Como (por el supuesto anterior) podemos
      inundar $G$ en tiempo a lo más \code{diam(G) - 1} [Si suponemos
      que esto es cierto para \code{diam(G)} resulta que el diámetro
      aumenta, esto va en contra de la consideración de conservar el
      diámetro de $G$ para $G'$].

      Así, supongamos sin pérdida de generalidad que, $p_j$ recibe el
      mensaje \code{M} en tiempo \code{diam(G) - 1} (desde algún proceso
      distinguido como líder, según el algoritmo \code{flooding}). Como
      $p_i$ y $p_j$ son vecinos, distribuir \code{M} toma tiempo $1$.
      Entonces la gráfica se inunda, para cada proceso, en tiempo a lo más
      \code{(diam(G) - 1) + (1) = diam(G)}.
      
      \item \code{diam(G') = diam(G) + 1}. Sabemos que $p_i$ tiene algún
      vecino $p_j$ (por conexidad de $G'$). Supongamos, sin pérdida de
      generalidad, que $p_j$ recibe el mensaje \code{M} en tiempo \code{diam(G)}.
      Así, transmitir \code{M} de $p_j$ a $p_i$ toma tiempo $1$, luego
      transmitir \code{M} desde un proceso distinguido como líder hacia
      $p_i$ toma tiempo \code{diam(G) + 1}\footnote{El tiempo de transmitirlo
      hasta $p_j$ más el tiempo de transmitirlo de $p_j$ a $p_i$.}.
      \end{enumerate}
      \hfill $\blacksquare$
\end{itemize}
