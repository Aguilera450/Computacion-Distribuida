\newpage\textbf{\textcolor{MidnightBlue}{1.}}
Considere el siguiente mecanismo de elección de líder vagamente monárquico para
un sistema de paso de mensajes asíncrono con fallas de tipo paro. Cada proceso tiene
acceso a un oráculo que inicia con el valor $0$ y puede incrementar sobre el tiempo.
El oráculo garantiza:
\begin{enumerate}[a)]
%%%%%%%%%%%%%%%%%       Inciso A        %%%%%%%%%%%%%%%%%
\item No hay dos procesos que vean el mismo valor distinto de cero.
%%%%%%%%%%%%%%%%%       Inciso B        %%%%%%%%%%%%%%%%%
\item Eventualmente algunos procesos no fallidos se les asigna un valor fijo que es
mayor que los valores de todos los demás procesos por el resto de la ejecución.
\end{enumerate}

En función del número de procesos $n$, ¿Cuál es el mayor número 
de fallas de tipo paro $f$ para el cuál es posible resolver consenso utilizando este oráculo?\\

Recordamos que la razón por la cual no es posible llegar a consenso 
en un sistema asíncrono con fallas de tipo paro es que los 
procesos no son capaces de  distinguir si un proceso ha fallado o 
solo es lento.

Si el proceso es capaz de preguntarle al oraculo si cierto proceso esta con 
vida ya que solo los procesos no fallidos tienen asignado un valor fijo 
diferente y mayor que los valores de todos lo demás procesos por el resto de la 
ejecución, entonces ya es posible resolver el problema de elección de líder,
sin embargo, solo algunos procesos conoceran este valor a pesar de estar con 
vida. Entonces a pesar de que gracias al oraculo algunos procesos conozcan un 
valor que cumple con las caracteristicas anteriores no es posible
solucionar la elección de líder.
