\newline
\textbf{\textcolor{MidnightBlue}{3.}}
Supongamos que no tenemos suficientes recursos para equipar todas nuestras máquinas
con detectores de fallos. En su lugar, ordenamos detectores de fallos eventualmente
fuertes para $k$ máquinas y las restantes $n-k$ máquinas tienen detectores de fallos
fake que nunca sospechan de nadie. La elección de cuales máquinas obtienen el
detector 
de fallos real y cuales obtienen los falsos, está bajo el control de un adversario.

Esto significa que todo proceso fallido es eventualmente puesto bajo sospecha de
manera permanente por todo procesos no fallido con un detector de fallos real, y hay
al menos un proceso no fallido que eventualmente no es puesto bajo sospecha de forma
permanente por nadie. Llamemos al detector de fallas resultante $\Diamond S_k$.\\

Sea $f$ el número actual de fallas. ¿Bajo que condiciones de $f$, $k$ y $n$ se puede
resolver el consenso en el modelo usual de paso de mensajes asíncrono determinista
usando $\Diamond S_k$? \newline

$\rhd$ \textbf{Solución:} Sabemos por definición de consenso con \code{$\Delta$S} que
$f < \frac{n}{2}$ es un condición necesaria para poder llegar a un consenso de manera
consistente. Ahora, sabemos que solo podemos equipar $k$ máquinas con nuestros detectores
de fallos y $n - k$ son máquinas con detectores de fallos \textit{fake}, entonces
¿qué tan grande es $k$ y $n - k$ para que aún podamos llegar a un consenso?

Supongamos sin pérdida de generalidad que existen $f = \frac{n}{2} - 1$, entonces nuestras
$k$-máquinas con detectores de fallos deben poder lidiar con estos fallos y el número
suiciente de estas es $k = n - 1$, dejando así a $n - k = n - (n - 1) ) = 1$. Si el número
de fallos en nuestra cota disminuye en $m$, entonces $k = (n - 1) - m$ y $n - k = m + 1$.

\hfill $\lhd$
