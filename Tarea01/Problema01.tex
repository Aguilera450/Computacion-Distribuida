%%%%%%%%%%% Aquí va la solución al problema 1.
\newpage
1. (2 puntos) ¿Cuáles son las principales diferencias entre
un sistema distribuido, un sistema concurrente y un sistema
paralelo? Argumenta detalladamente.\\

Un sistema distribuido permite que una colección de computadoras independientes,
que pueden o no estar ubicadas en distintos lugares, trabajar en conjunto para
lograr una tarea. Utilizan protocolos especiales y paso de mensajes para
coordinarse. La finalidad es es lograr solucionar problemas repartiendo el
trabajo entre cada uno de los agentes del sistema.\\

En un sistema concurrente diferentes procesos o equipos se intercambian ya
acceden a un mismo recurso durante periodos de tiempo separados de forma
ordenada y nunca juntos. Este acceso puede ser tan rápido que desde el punto
de vista del usuario pareciera que múltiples procesos acceden al recurso al
mismo tiempo.\\

En un sistema paralelo se hace uso de 2 o más procesadores para resolver una
tarea. Se basa en el principio según el cual, algunas tareas se pueden dividir
en partes más pequeñas que pueden resolverse simultáneamente.
