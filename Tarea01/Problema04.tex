%%%%%%%%%%% Aquí va la solución al problema 1.
\newpage
\textbf{\textcolor{MidnightBlue}{4.}} Considera un sistema
distribuido con $n \geq 2$ procesos, $p_1, p_2, \dotsm , p_n$,
en el que la gráfica de comunicación es la completa $K_n$.
El sistema es síncrono pero la comunicación no es confiable;
sea P el conjunto de todos los procesos que envían mensajes en
el tiempo $d$; entonces, hay dos posibilidades, todos los
mensajes de P llegan a su destino en el tiempo $d + 1$, o uno
de ellos se pierde y nunca llega a sus destino y los otros
en $P$ si llegan en el tiempo $d + 1$.
\newline

Considera un algoritmo A en el que cada proceso $p_i$ tiene como
entrada un identificador $ID_i$, que es un número natural (diferente
al de los demás), y cada proceso $p_i$ simplemente envía su $ID_i$
a los otros $n - 1$ procesos.
\newline

Dibuja cuales son todos los estados \textit{globales} posibles
(mundos posibles) en el tiempo 1 (los procesos mandan sus mensajes
en el tiempo 0). En cada estado global, especifica el estado
\textit{local} de cada proceso, es decir, la información que cada
proceso tiene en ese estado global; y entre cada par de estados
globales pinta una arista con los procesos que no pueden \textit{distinguir}
entre esos estados. ¿Es posible que cada proceso elija consistentemente
uno de los $ID$s de entre los que recibió de forma tal que en cada estado
global todos los procesos eligen el mismo $ID$? Argumenta tu respuesta.
\newline

\textbf{\textit{Solución:}}
Supongamos que $n = 3$, entonces la representación gráfica de lo requerido
se vería como el siguiente gráfico:


\begin{figure}[ht!]
  \centering
  \begin{tikzpicture}
  %%%%%%%%%%%%%%%%%% Nodos p_i
    \node (1)  [vertex, label=270:$p_1$] at (0,0){};
    \node (2)  [redV, label=270:$p_2$] at (2,0){};
    \node (3)  [vertex, label=90:$p_3$] at (1,1.2){};
    
    \node (4)  [vertex, label=90:$p_3$] at  (9,1.2){};
    \node (5)  [blueV, label=270:$p_1$] at (8,0){};
    \node (6)  [vertex, label=270:$p_2$] at (10,0){};

    \node (7) [vertex, label=90:$p_1$] at (4,-2){};
    \node (8) [vertex, label=90:$p_2$] at (6,-2){};
    \node (9) [vertex, label=270:$p_3$] at (5,-3.2){};

    \node (10) [blackV, label=90:$p_{3}$] at (5, -5.2){};
    \node (11) [vertex, label=270:$p_{1}$] at (4, -6.4){};
    \node (12) [vertex, label=270:$p_{2}$] at (6, -6.4){};

    %%%%%%%%%%%%%%%%%%%%%% Aristas
    \foreach \i in {1,2}
    \draw [edge] let \n1={3} in (\i) to (\n1);
    \foreach \i in {5,6}
    \draw [edge] let \n1={4} in (\i) to (\n1);
    \foreach \i in {7,8}
    \draw [edge] let \n1={9} in (\i) to (\n1);
    \foreach \i in {11,12}
    \draw [edge] let \n1={10} in (\i) to (\n1);
    \draw [edge]  (1) to (2);
    \draw [edge]  (5) to (6);
    \draw [edge]  (7) to (8);
    \draw [edge]  (11) to (12);
    
    \node (L) at (-1,2){\textbf{P}};
    \draw [edge]  (-2,3) to (12,3);
    \draw [edge]  (-2,-8) to (12,-8);
    \draw [edge]  (-2,3) to (-2,-8);
    \draw [edge]  (12,3) to (12,-8);
  \end{tikzpicture}
\end{figure}

donde, a cada $p_i$ le corresponde un $ID_i$ de tal manera que
para cada estado global se identifica si todos sus procesos fueron
recibidos de manera exitosa, en caso de no ser así se colorean.
Así,
\begin{enumerate}
\item 
\item 
\item 
\end{enumerate}
\hfill $\lhd$
